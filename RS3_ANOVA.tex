\documentclass[]{article}
\usepackage{lmodern}
\usepackage{amssymb,amsmath}
\usepackage{ifxetex,ifluatex}
\usepackage{fixltx2e} % provides \textsubscript
\ifnum 0\ifxetex 1\fi\ifluatex 1\fi=0 % if pdftex
  \usepackage[T1]{fontenc}
  \usepackage[utf8]{inputenc}
\else % if luatex or xelatex
  \ifxetex
    \usepackage{mathspec}
  \else
    \usepackage{fontspec}
  \fi
  \defaultfontfeatures{Ligatures=TeX,Scale=MatchLowercase}
\fi
% use upquote if available, for straight quotes in verbatim environments
\IfFileExists{upquote.sty}{\usepackage{upquote}}{}
% use microtype if available
\IfFileExists{microtype.sty}{%
\usepackage{microtype}
\UseMicrotypeSet[protrusion]{basicmath} % disable protrusion for tt fonts
}{}
\usepackage[margin=1in]{geometry}
\usepackage{hyperref}
\hypersetup{unicode=true,
            pdftitle={2x2 repeated measures ANOVA with RS3 data},
            pdfborder={0 0 0},
            breaklinks=true}
\urlstyle{same}  % don't use monospace font for urls
\usepackage{color}
\usepackage{fancyvrb}
\newcommand{\VerbBar}{|}
\newcommand{\VERB}{\Verb[commandchars=\\\{\}]}
\DefineVerbatimEnvironment{Highlighting}{Verbatim}{commandchars=\\\{\}}
% Add ',fontsize=\small' for more characters per line
\usepackage{framed}
\definecolor{shadecolor}{RGB}{248,248,248}
\newenvironment{Shaded}{\begin{snugshade}}{\end{snugshade}}
\newcommand{\KeywordTok}[1]{\textcolor[rgb]{0.13,0.29,0.53}{\textbf{#1}}}
\newcommand{\DataTypeTok}[1]{\textcolor[rgb]{0.13,0.29,0.53}{#1}}
\newcommand{\DecValTok}[1]{\textcolor[rgb]{0.00,0.00,0.81}{#1}}
\newcommand{\BaseNTok}[1]{\textcolor[rgb]{0.00,0.00,0.81}{#1}}
\newcommand{\FloatTok}[1]{\textcolor[rgb]{0.00,0.00,0.81}{#1}}
\newcommand{\ConstantTok}[1]{\textcolor[rgb]{0.00,0.00,0.00}{#1}}
\newcommand{\CharTok}[1]{\textcolor[rgb]{0.31,0.60,0.02}{#1}}
\newcommand{\SpecialCharTok}[1]{\textcolor[rgb]{0.00,0.00,0.00}{#1}}
\newcommand{\StringTok}[1]{\textcolor[rgb]{0.31,0.60,0.02}{#1}}
\newcommand{\VerbatimStringTok}[1]{\textcolor[rgb]{0.31,0.60,0.02}{#1}}
\newcommand{\SpecialStringTok}[1]{\textcolor[rgb]{0.31,0.60,0.02}{#1}}
\newcommand{\ImportTok}[1]{#1}
\newcommand{\CommentTok}[1]{\textcolor[rgb]{0.56,0.35,0.01}{\textit{#1}}}
\newcommand{\DocumentationTok}[1]{\textcolor[rgb]{0.56,0.35,0.01}{\textbf{\textit{#1}}}}
\newcommand{\AnnotationTok}[1]{\textcolor[rgb]{0.56,0.35,0.01}{\textbf{\textit{#1}}}}
\newcommand{\CommentVarTok}[1]{\textcolor[rgb]{0.56,0.35,0.01}{\textbf{\textit{#1}}}}
\newcommand{\OtherTok}[1]{\textcolor[rgb]{0.56,0.35,0.01}{#1}}
\newcommand{\FunctionTok}[1]{\textcolor[rgb]{0.00,0.00,0.00}{#1}}
\newcommand{\VariableTok}[1]{\textcolor[rgb]{0.00,0.00,0.00}{#1}}
\newcommand{\ControlFlowTok}[1]{\textcolor[rgb]{0.13,0.29,0.53}{\textbf{#1}}}
\newcommand{\OperatorTok}[1]{\textcolor[rgb]{0.81,0.36,0.00}{\textbf{#1}}}
\newcommand{\BuiltInTok}[1]{#1}
\newcommand{\ExtensionTok}[1]{#1}
\newcommand{\PreprocessorTok}[1]{\textcolor[rgb]{0.56,0.35,0.01}{\textit{#1}}}
\newcommand{\AttributeTok}[1]{\textcolor[rgb]{0.77,0.63,0.00}{#1}}
\newcommand{\RegionMarkerTok}[1]{#1}
\newcommand{\InformationTok}[1]{\textcolor[rgb]{0.56,0.35,0.01}{\textbf{\textit{#1}}}}
\newcommand{\WarningTok}[1]{\textcolor[rgb]{0.56,0.35,0.01}{\textbf{\textit{#1}}}}
\newcommand{\AlertTok}[1]{\textcolor[rgb]{0.94,0.16,0.16}{#1}}
\newcommand{\ErrorTok}[1]{\textcolor[rgb]{0.64,0.00,0.00}{\textbf{#1}}}
\newcommand{\NormalTok}[1]{#1}
\usepackage{graphicx,grffile}
\makeatletter
\def\maxwidth{\ifdim\Gin@nat@width>\linewidth\linewidth\else\Gin@nat@width\fi}
\def\maxheight{\ifdim\Gin@nat@height>\textheight\textheight\else\Gin@nat@height\fi}
\makeatother
% Scale images if necessary, so that they will not overflow the page
% margins by default, and it is still possible to overwrite the defaults
% using explicit options in \includegraphics[width, height, ...]{}
\setkeys{Gin}{width=\maxwidth,height=\maxheight,keepaspectratio}
\IfFileExists{parskip.sty}{%
\usepackage{parskip}
}{% else
\setlength{\parindent}{0pt}
\setlength{\parskip}{6pt plus 2pt minus 1pt}
}
\setlength{\emergencystretch}{3em}  % prevent overfull lines
\providecommand{\tightlist}{%
  \setlength{\itemsep}{0pt}\setlength{\parskip}{0pt}}
\setcounter{secnumdepth}{0}
% Redefines (sub)paragraphs to behave more like sections
\ifx\paragraph\undefined\else
\let\oldparagraph\paragraph
\renewcommand{\paragraph}[1]{\oldparagraph{#1}\mbox{}}
\fi
\ifx\subparagraph\undefined\else
\let\oldsubparagraph\subparagraph
\renewcommand{\subparagraph}[1]{\oldsubparagraph{#1}\mbox{}}
\fi

%%% Use protect on footnotes to avoid problems with footnotes in titles
\let\rmarkdownfootnote\footnote%
\def\footnote{\protect\rmarkdownfootnote}

%%% Change title format to be more compact
\usepackage{titling}

% Create subtitle command for use in maketitle
\providecommand{\subtitle}[1]{
  \posttitle{
    \begin{center}\large#1\end{center}
    }
}

\setlength{\droptitle}{-2em}

  \title{2x2 repeated measures ANOVA with RS3 data}
    \pretitle{\vspace{\droptitle}\centering\huge}
  \posttitle{\par}
    \author{}
    \preauthor{}\postauthor{}
    \date{}
    \predate{}\postdate{}
  

\begin{document}
\maketitle

The following markdown analyses data from the RS3 in R. The results
(main effects, interactions and follow up t-tests) are the same as JASP
and SPSS.

\begin{Shaded}
\begin{Highlighting}[]
\CommentTok{# rm(list = ls()) # clears environment}

\KeywordTok{library}\NormalTok{(ggplot2) }\CommentTok{# plotting interaction plots}
\KeywordTok{library}\NormalTok{(Rmisc) }\CommentTok{# for confidence intervals on the interaction plots}
\KeywordTok{library}\NormalTok{(tidyr) }\CommentTok{# manipulating dataframes}
\KeywordTok{library}\NormalTok{(dplyr) }\CommentTok{# manipulating dataframes}
\KeywordTok{library}\NormalTok{(ez) }\CommentTok{# allows me to compute repeated ANOVA that matches JASP output}
\KeywordTok{library}\NormalTok{(car) }\CommentTok{# allows me to use Anova() function to get summary of main effects and interactions}
\KeywordTok{library}\NormalTok{(sjstats) }\CommentTok{# computes partial eta squared effect size}
\KeywordTok{library}\NormalTok{(effsize) }\CommentTok{# computes cohen's D for follow up t-tests}
\KeywordTok{library}\NormalTok{(apaTables) }\CommentTok{# for apa tables}
\end{Highlighting}
\end{Shaded}

Here I am loading the packages I'll need for the analysis. I have put a
note about why each one is needed. There are many ways to compute ANOVAs
in R and there are many different packages you can use to do so. However
I have found that the ``ez'' packages works best for a repeated measures
design. I explain in the following markdown.

\begin{Shaded}
\begin{Highlighting}[]
\CommentTok{# work computer working directory}
\KeywordTok{setwd}\NormalTok{(}\StringTok{"C:/Users/pscmgo/OneDrive for Business/PhD/Project/Experiment_Code/teachingcode"}\NormalTok{)}

\CommentTok{# load datset}
\NormalTok{temp =}\StringTok{ }\KeywordTok{list.files}\NormalTok{(}\DataTypeTok{pattern =} \KeywordTok{c}\NormalTok{(}\StringTok{"RS3_dataset"}\NormalTok{, }\StringTok{"*.csv"}\NormalTok{)) }\CommentTok{# list all CSV files in the directory}
\NormalTok{myfiles =}\StringTok{ }\KeywordTok{lapply}\NormalTok{(temp, read.csv) }\CommentTok{# read these CSV in the directory}
\NormalTok{workingdata <-}\StringTok{ }\KeywordTok{do.call}\NormalTok{(rbind.data.frame, myfiles) }\CommentTok{# convert and combine the CSV files into dataframe}
\end{Highlighting}
\end{Shaded}

Here I am grabbing the data from my working directory

\begin{Shaded}
\begin{Highlighting}[]
\CommentTok{# Here I'm generating a participant ID, this will be important later}
\NormalTok{workingdata <-}\StringTok{ }\NormalTok{workingdata }\OperatorTok
\StringTok{  }\KeywordTok{mutate}\NormalTok{(}\DataTypeTok{ppid =} \KeywordTok{row_number}\NormalTok{())}

\CommentTok{# computing histograms and normality curves for each condition}
\KeywordTok{ggplot}\NormalTok{(workingdata }\OperatorTok
\StringTok{         }\KeywordTok{gather}\NormalTok{(}\StringTok{"NoAS_Ver"}\NormalTok{, }\StringTok{"NoAS_VerDemo"}\NormalTok{, }\StringTok{"AS_Ver"}\NormalTok{, }\StringTok{"AS_VerDemo"}\NormalTok{, }\DataTypeTok{key =} \StringTok{"condition"}\NormalTok{, }\DataTypeTok{value =} \StringTok{"correctpairs"}\NormalTok{), }\KeywordTok{aes}\NormalTok{(}\DataTypeTok{x =}\NormalTok{ correctpairs)) }\OperatorTok{+}
\StringTok{  }\KeywordTok{geom_histogram}\NormalTok{(}\KeywordTok{aes}\NormalTok{(}\DataTypeTok{y =}\NormalTok{ ..density..)) }\OperatorTok{+}
\StringTok{  }\KeywordTok{stat_function}\NormalTok{(}
        \DataTypeTok{fun =}\NormalTok{ dnorm, }
        \DataTypeTok{args =} \KeywordTok{with}\NormalTok{(workingdata }\OperatorTok
\StringTok{                      }\KeywordTok{gather}\NormalTok{(}\StringTok{"NoAS_Ver"}\NormalTok{, }\StringTok{"NoAS_VerDemo"}\NormalTok{, }\StringTok{"AS_Ver"}\NormalTok{, }\StringTok{"AS_VerDemo"}\NormalTok{, }\DataTypeTok{key =} \StringTok{"condition"}\NormalTok{, }\DataTypeTok{value =} \StringTok{"correctpairs"}\NormalTok{), }\KeywordTok{c}\NormalTok{(}\DataTypeTok{mean =} \KeywordTok{mean}\NormalTok{(correctpairs), }\DataTypeTok{sd =} \KeywordTok{sd}\NormalTok{(correctpairs)))) }\OperatorTok{+}\StringTok{ }
\StringTok{  }\KeywordTok{facet_wrap}\NormalTok{( }\OperatorTok{~}\StringTok{ }\NormalTok{condition)}
\end{Highlighting}
\end{Shaded}

\begin{verbatim}
## `stat_bin()` using `bins = 30`. Pick better value with `binwidth`.
\end{verbatim}

\includegraphics{RS3_ANOVA_files/figure-latex/descriptive statistics-1.pdf}

\begin{Shaded}
\begin{Highlighting}[]
\CommentTok{# descriptives for age}
\KeywordTok{range}\NormalTok{(workingdata}\OperatorTok{$}\NormalTok{Age)}
\end{Highlighting}
\end{Shaded}

\begin{verbatim}
## [1] 18 35
\end{verbatim}

\begin{Shaded}
\begin{Highlighting}[]
\KeywordTok{mean}\NormalTok{(workingdata}\OperatorTok{$}\NormalTok{Age)}
\end{Highlighting}
\end{Shaded}

\begin{verbatim}
## [1] 19.92584
\end{verbatim}

\begin{Shaded}
\begin{Highlighting}[]
\KeywordTok{sd}\NormalTok{(workingdata}\OperatorTok{$}\NormalTok{Age)}
\end{Highlighting}
\end{Shaded}

\begin{verbatim}
## [1] 1.886744
\end{verbatim}

Firstly I'm creating a new variable ``ppid'' to indicate the participant
number. This will be important when the ANOVA in R attempts to calculate
the degrees of freedom.

Histograms are plotted with normal distributions to show how the data is
distributed. In this case, things seem fairly normally distributed.

\begin{Shaded}
\begin{Highlighting}[]
\CommentTok{# spliting conditions into factors and their levels}
\NormalTok{split_data <-}\StringTok{ }\NormalTok{workingdata }\OperatorTok
\StringTok{  }\KeywordTok{gather}\NormalTok{(}\StringTok{"NoAS_Ver"}\NormalTok{, }\StringTok{"NoAS_VerDemo"}\NormalTok{, }\StringTok{"AS_Ver"}\NormalTok{, }\StringTok{"AS_VerDemo"}\NormalTok{, }\DataTypeTok{key =} \StringTok{"condition"}\NormalTok{, }\DataTypeTok{value =} \StringTok{"correctpairs"}\NormalTok{) }\OperatorTok
\StringTok{  }\KeywordTok{separate}\NormalTok{(condition, }\DataTypeTok{into =} \KeywordTok{c}\NormalTok{(}\StringTok{"task"}\NormalTok{, }\StringTok{"presentation"}\NormalTok{))}

\KeywordTok{head}\NormalTok{(split_data, }\DataTypeTok{n =} \DecValTok{5}\NormalTok{)}
\end{Highlighting}
\end{Shaded}

\begin{verbatim}
##   Age Gender ppid task presentation correctpairs
## 1  20      1    1 NoAS          Ver        2.125
## 2  20      2    2 NoAS          Ver        3.250
## 3  20      2    3 NoAS          Ver        1.875
## 4  19      1    4 NoAS          Ver        2.625
## 5  19      2    5 NoAS          Ver        3.000
\end{verbatim}

\begin{Shaded}
\begin{Highlighting}[]
\KeywordTok{tail}\NormalTok{(split_data, }\DataTypeTok{n =} \DecValTok{5}\NormalTok{)}
\end{Highlighting}
\end{Shaded}

\begin{verbatim}
##      Age Gender ppid task presentation correctpairs
## 1776  19      1  441   AS      VerDemo        3.125
## 1777  19      2  442   AS      VerDemo        2.120
## 1778  19      2  443   AS      VerDemo        1.500
## 1779  19      2  444   AS      VerDemo        3.125
## 1780  22      1  445   AS      VerDemo        3.380
\end{verbatim}

\begin{Shaded}
\begin{Highlighting}[]
\CommentTok{# sets our factors as a factor data types}
\NormalTok{split_data}\OperatorTok{$}\NormalTok{task <-}\StringTok{ }\KeywordTok{as.factor}\NormalTok{(split_data}\OperatorTok{$}\NormalTok{task)}
\NormalTok{split_data}\OperatorTok{$}\NormalTok{presentation <-}\StringTok{ }\KeywordTok{as.factor}\NormalTok{(split_data}\OperatorTok{$}\NormalTok{presentation)}
\NormalTok{split_data}\OperatorTok{$}\NormalTok{ppid <-}\StringTok{ }\KeywordTok{as.factor}\NormalTok{(split_data}\OperatorTok{$}\NormalTok{ppid)}
\end{Highlighting}
\end{Shaded}

In JASP, each column represents a condition. In R, I need the columns to
refer to the different factors, with each row being an observation. I
use the gather function to create a condition and value column. I then
use the separate function to seperate the conditions into ``task'' and
``presentation''. The head and tail function demonstrate what this data
frame looks like now.

I then ensure the factors in this design are set to factor data types.

\begin{Shaded}
\begin{Highlighting}[]
\CommentTok{# running the ANOVA this way generates the wrong f values and sums of squares}
\NormalTok{result_wrong <-}\StringTok{ }\KeywordTok{aov}\NormalTok{(correctpairs }\OperatorTok{~}\StringTok{ }\NormalTok{task }\OperatorTok{*}\StringTok{ }\NormalTok{presentation, }\DataTypeTok{data =}\NormalTok{ split_data)}
\KeywordTok{Anova}\NormalTok{(result_wrong, }\DataTypeTok{type =} \DecValTok{3}\NormalTok{)}
\end{Highlighting}
\end{Shaded}

\begin{verbatim}
## Anova Table (Type III tests)
## 
## Response: correctpairs
##                    Sum Sq   Df  F value    Pr(>F)    
## (Intercept)       1892.71    1 5867.679 < 2.2e-16 ***
## task                80.72    1  250.241 < 2.2e-16 ***
## presentation       107.40    1  332.945 < 2.2e-16 ***
## task:presentation   18.50    1   57.343 5.852e-14 ***
## Residuals          572.88 1776                       
## ---
## Signif. codes:  0 '***' 0.001 '**' 0.01 '*' 0.05 '.' 0.1 ' ' 1
\end{verbatim}

This is how I first tried to run the ANOVA. However as you can see, the
sum of squares, f values and degrees of freedom (DF) do not match JASP
and SPSS output (despite using type 3 sum of squares). After many nights
of searching, I realised the problem is the DFs.

\begin{Shaded}
\begin{Highlighting}[]
\KeywordTok{nrow}\NormalTok{(split_data)}
\end{Highlighting}
\end{Shaded}

\begin{verbatim}
## [1] 1780
\end{verbatim}

The DFs are the number of observations that are allowed to vary in the
statistical calculation and are calculated as the N - 1. However because
I have arranged the dataframe so that each column is a factor, the
dataframe is telling the ANOVA that I have 1780 observations. I actually
have 445 (the number of participants).

\begin{Shaded}
\begin{Highlighting}[]
\CommentTok{# running the ANOVA this way generates the correc f values and sum of sqaures}
\NormalTok{result_correct <-}\StringTok{ }\KeywordTok{ezANOVA}\NormalTok{(split_data, }\DataTypeTok{dv =}\NormalTok{ correctpairs, }\DataTypeTok{wid =}\NormalTok{ ppid, }\DataTypeTok{within =}\NormalTok{ .(task, presentation), }\DataTypeTok{type =} \DecValTok{3}\NormalTok{, }\DataTypeTok{detailed =} \OtherTok{TRUE}\NormalTok{, }\DataTypeTok{return_aov =} \OtherTok{TRUE}\NormalTok{)}
\NormalTok{result_correct}
\end{Highlighting}
\end{Shaded}

\begin{verbatim}
## $ANOVA
##              Effect DFn DFd         SSn       SSd           F            p
## 1       (Intercept)   1 444 12115.72639 287.90366 18684.66151 0.000000e+00
## 2              task   1 444    70.64509  99.96981   313.75889 1.703071e-53
## 3      presentation   1 444   107.22704  84.41543   563.98226 4.527024e-81
## 4 task:presentation   1 444    18.49669 100.58709    81.64596 5.038954e-18
##   p<.05        ges
## 1     * 0.95485113
## 2     * 0.10977898
## 3     * 0.15766294
## 4     * 0.03127755
## 
## $aov
## 
## Call:
## aov(formula = formula(aov_formula), data = data)
## 
## Grand Mean: 2.608944
## 
## Stratum 1: ppid
## 
## Terms:
##                 Residuals
## Sum of Squares   287.9037
## Deg. of Freedom       444
## 
## Residual standard error: 0.8052526
## 
## Stratum 2: ppid:task
## 
## Terms:
##                     task Residuals
## Sum of Squares  70.64509  99.96981
## Deg. of Freedom        1       444
## 
## Residual standard error: 0.4745074
## 1 out of 2 effects not estimable
## Estimated effects are balanced
## 
## Stratum 3: ppid:presentation
## 
## Terms:
##                 presentation Residuals
## Sum of Squares     107.22704  84.41543
## Deg. of Freedom            1       444
## 
## Residual standard error: 0.4360331
## 1 out of 2 effects not estimable
## Estimated effects are balanced
## 
## Stratum 4: ppid:task:presentation
## 
## Terms:
##                 task:presentation Residuals
## Sum of Squares           18.49669 100.58709
## Deg. of Freedom                 1       444
## 
## Residual standard error: 0.4759701
## Estimated effects are balanced
\end{verbatim}

If I use the ezANOVA function, I can specify the number of cases I
actually have with the ``wid'' arugument. I assign this the ppid
variable made earlier. This means that the DFs can be correctly
calculated, and now the sum of squares (SSn), F value and DFs match the
JASP and SPSS output.

Importantly, I input ``return\_aov = TRUE''. This is important for the
effect sizes.

\begin{Shaded}
\begin{Highlighting}[]
\CommentTok{# effect sizes for the correct ANOVA method}
\KeywordTok{eta_sq}\NormalTok{(result_correct}\OperatorTok{$}\NormalTok{aov, }\DataTypeTok{partial =} \OtherTok{TRUE}\NormalTok{)}
\end{Highlighting}
\end{Shaded}

\begin{verbatim}
##                term partial.etasq                stratum
## 1         Residuals         0.741                   ppid
## 2              task         0.413              ppid:task
## 3         Residuals         0.498              ppid:task
## 4      presentation         0.516      ppid:presentation
## 5         Residuals         0.456      ppid:presentation
## 6 task:presentation         0.155 ppid:task:presentation
\end{verbatim}

The effect size function only takes ``aov'' objects. Because I returned
an ``aov'' object from the ezANOVA, I can use this to calculate the
effect size.

\begin{Shaded}
\begin{Highlighting}[]
\KeywordTok{model.tables}\NormalTok{(result_correct}\OperatorTok{$}\NormalTok{aov, }\StringTok{"mean"}\NormalTok{, }\DataTypeTok{se =} \OtherTok{TRUE}\NormalTok{)}
\end{Highlighting}
\end{Shaded}

\begin{verbatim}
## Warning in model.tables.aovlist(result_correct$aov, "mean", se = TRUE): SEs
## for type 'means' are not yet implemented
\end{verbatim}

\begin{verbatim}
## Tables of means
## Grand mean
##          
## 2.608944 
## 
##  task 
## task
##     AS   NoAS 
## 2.4097 2.8082 
## 
##  presentation 
## presentation
##     Ver VerDemo 
##  2.3635  2.8544 
## 
##  task:presentation 
##       presentation
## task   Ver    VerDemo
##   AS   2.0623 2.7571 
##   NoAS 2.6647 2.9517
\end{verbatim}

I then also use the ``aov'' object from the ezANOVA to calculate the
estimated marginal means for the main effects. Again, these match the
ones from the SPSS output.

\begin{Shaded}
\begin{Highlighting}[]
\CommentTok{# computes standard error and 95% confidence intervals}
\NormalTok{CIerror_bar <-}\StringTok{ }\KeywordTok{summarySE}\NormalTok{(split_data, }\DataTypeTok{measurevar =} \StringTok{"correctpairs"}\NormalTok{, }\DataTypeTok{groupvars =} \KeywordTok{c}\NormalTok{(}\StringTok{"task"}\NormalTok{, }\StringTok{"presentation"}\NormalTok{))}

\KeywordTok{ggplot}\NormalTok{(}\DataTypeTok{data =}\NormalTok{ CIerror_bar, }\KeywordTok{aes}\NormalTok{(}\DataTypeTok{x =}\NormalTok{ presentation, }\DataTypeTok{color =}\NormalTok{ task, }\DataTypeTok{group =}\NormalTok{ task, }\DataTypeTok{y =}\NormalTok{ correctpairs)) }\OperatorTok{+}
\StringTok{         }\KeywordTok{stat_summary}\NormalTok{(}\DataTypeTok{fun.y =}\NormalTok{ mean, }\DataTypeTok{geom =} \StringTok{"point"}\NormalTok{) }\OperatorTok{+}
\StringTok{         }\KeywordTok{stat_summary}\NormalTok{(}\DataTypeTok{fun.y =}\NormalTok{ mean, }\DataTypeTok{geom =} \StringTok{"line"}\NormalTok{) }\OperatorTok{+}
\StringTok{  }\KeywordTok{geom_errorbar}\NormalTok{(}\KeywordTok{aes}\NormalTok{(}\DataTypeTok{ymin =}\NormalTok{ correctpairs }\OperatorTok{-}\StringTok{ }\NormalTok{se, }\DataTypeTok{ymax =}\NormalTok{ correctpairs }\OperatorTok{+}\StringTok{ }\NormalTok{se), }\DataTypeTok{width =}\NormalTok{ .}\DecValTok{1}\NormalTok{)}
\end{Highlighting}
\end{Shaded}

\includegraphics{RS3_ANOVA_files/figure-latex/interaction plot-1.pdf}

Here I have an interaction plot that visualises the interaction. I have
also plotted 95\% confidence intervals around each mean.

\begin{Shaded}
\begin{Highlighting}[]
\NormalTok{workingdata <-}\StringTok{ }\NormalTok{split_data }\OperatorTok
\StringTok{  }\KeywordTok{unite}\NormalTok{(}\StringTok{"condition"}\NormalTok{, task, presentation, }\DataTypeTok{sep =} \StringTok{"_"}\NormalTok{) }\OperatorTok
\StringTok{  }\KeywordTok{spread}\NormalTok{(condition, correctpairs)}
\end{Highlighting}
\end{Shaded}

To compute the t-tests, I re-order the dataframe so that each condition
is a column.

\begin{Shaded}
\begin{Highlighting}[]
\CommentTok{# t-test and effect size for AS and no AS in the verbal condition}
\KeywordTok{t.test}\NormalTok{(workingdata}\OperatorTok{$}\NormalTok{NoAS_Ver, workingdata}\OperatorTok{$}\NormalTok{AS_Ver, }\DataTypeTok{paired =} \OtherTok{TRUE}\NormalTok{)}
\end{Highlighting}
\end{Shaded}

\begin{verbatim}
## 
##  Paired t-test
## 
## data:  workingdata$NoAS_Ver and workingdata$AS_Ver
## t = 18.206, df = 444, p-value < 2.2e-16
## alternative hypothesis: true difference in means is not equal to 0
## 95 percent confidence interval:
##  0.5372949 0.6673343
## sample estimates:
## mean of the differences 
##               0.6023146
\end{verbatim}

\begin{Shaded}
\begin{Highlighting}[]
\KeywordTok{cohen.d}\NormalTok{(workingdata}\OperatorTok{$}\NormalTok{NoAS_Ver, workingdata}\OperatorTok{$}\NormalTok{AS_Ver, }\DataTypeTok{paired =} \OtherTok{TRUE}\NormalTok{)}
\end{Highlighting}
\end{Shaded}

\begin{verbatim}
## 
## Cohen's d
## 
## d estimate: 1.035294 (large)
## 95 percent confidence interval:
##     lower     upper 
## 0.8969765 1.1736106
\end{verbatim}

\begin{Shaded}
\begin{Highlighting}[]
\CommentTok{# t-test and effect size for AS and no AS in the verbal condition}
\KeywordTok{t.test}\NormalTok{(workingdata}\OperatorTok{$}\NormalTok{NoAS_VerDemo, workingdata}\OperatorTok{$}\NormalTok{AS_VerDemo, }\DataTypeTok{paired =} \OtherTok{TRUE}\NormalTok{)}
\end{Highlighting}
\end{Shaded}

\begin{verbatim}
## 
##  Paired t-test
## 
## data:  workingdata$NoAS_VerDemo and workingdata$AS_VerDemo
## t = 6.3608, df = 444, p-value = 5e-10
## alternative hypothesis: true difference in means is not equal to 0
## 95 percent confidence interval:
##  0.1344469 0.2546767
## sample estimates:
## mean of the differences 
##               0.1945618
\end{verbatim}

\begin{Shaded}
\begin{Highlighting}[]
\KeywordTok{cohen.d}\NormalTok{(workingdata}\OperatorTok{$}\NormalTok{NoAS_VerDemo, workingdata}\OperatorTok{$}\NormalTok{AS_VerDemo, }\DataTypeTok{paired =} \OtherTok{TRUE}\NormalTok{)}
\end{Highlighting}
\end{Shaded}

\begin{verbatim}
## 
## Cohen's d
## 
## d estimate: 0.3506109 (small)
## 95 percent confidence interval:
##     lower     upper 
## 0.2391533 0.4620685
\end{verbatim}

I can then compute t-tests and cohen's D effect sizes for AS and no AS
in the verbal and and verbal + demo presentation factor levels.

\begin{Shaded}
\begin{Highlighting}[]
\KeywordTok{setwd}\NormalTok{(}\StringTok{"C:/Users/pscmgo/OneDrive for Business/PhD/Project/Experiment_Code/teachingcode"}\NormalTok{)}

\KeywordTok{apa.2way.table}\NormalTok{(presentation, task, correctpairs, }\DataTypeTok{data =}\NormalTok{ split_data, }\DataTypeTok{filename =} \StringTok{"anova_means_table.doc"}\NormalTok{, }\DataTypeTok{table.number =} \DecValTok{1}\NormalTok{,}
  \DataTypeTok{show.conf.interval =} \OtherTok{TRUE}\NormalTok{, }\DataTypeTok{show.marginal.means =} \OtherTok{TRUE}\NormalTok{)}
\end{Highlighting}
\end{Shaded}

\begin{verbatim}
## 
## 
## Table 1 
## 
## Means and standard deviations for correctpairs as a function of a 2(presentation) X 2(task) design 
## 
##                  M     M_95%_CI   SD
##       task:AS                       
##  presentation                       
##           Ver 2.06 [2.01, 2.11] 0.54
##       VerDemo 2.76 [2.70, 2.81] 0.57
##                                     
##     task:NoAS                       
##  presentation                       
##           Ver 2.66 [2.61, 2.72] 0.61
##       VerDemo 2.95 [2.90, 3.00] 0.54
## 
## Note. M and SD represent mean and standard deviation, respectively. 
## LL and UL indicate the lower and upper limits of the 
## 95% confidence interval for the mean, respectively. 
## The confidence interval is a plausible range of population means 
## that could have created a sample mean (Cumming, 2014).
\end{verbatim}

\begin{Shaded}
\begin{Highlighting}[]
\KeywordTok{apa.ezANOVA.table}\NormalTok{(result_correct, }\DataTypeTok{table.title =} \StringTok{"Within subjects effects"}\NormalTok{, }\DataTypeTok{table.number =} \DecValTok{2}\NormalTok{, }\DataTypeTok{filename =} \StringTok{"anova_table.doc"}\NormalTok{)}
\end{Highlighting}
\end{Shaded}

\begin{verbatim}
## 
## 
## Table 2 
## 
## Within subjects effects 
## 
##            Predictor df_num df_den   SS_num SS_den        F    p ges
##          (Intercept)      1    444 12115.73 287.90 18684.66 .000 .95
##                 task      1    444    70.65  99.97   313.76 .000 .11
##         presentation      1    444   107.23  84.42   563.98 .000 .16
##  task x presentation      1    444    18.50 100.59    81.65 .000 .03
## 
## Note. df_num indicates degrees of freedom numerator. df_den indicates degrees of freedom denominator. 
## SS_num indicates sum of squares numerator. SS_den indicates sum of squares denominator. 
## ges indicates generalized eta-squared.
## 
\end{verbatim}

Finally, a nice little touch. The apaTables package allows me to save
the mean and SDs from the conditions and the ANOVA tables as an APA
formatted table in a word document. The package also includes functions
that do this for over tests such as regressions and one-way ANOVAs.


\end{document}
